\documentclass[preview,border=1pt]{standalone}
\usepackage{amsmath,amssymb}
\usepackage{charter}
\usepackage{indentfirst}
\usepackage[utf8]{vietnam}
\usepackage{longtable}
\usepackage{multirow}
\usepackage{mathrsfs}
\usepackage{tabvar}
\usepackage{ifpdf}
\begin{document}
	Sắp xếp thông tin ở cột I với cột II sau đây để xác định đúng tiến trình nhà Nguyễn kí các hiệp ước đầu hàng thực dân Pháp nửa sau thế kỉ XIX.\\
\begin{center}
	%\includegraphics[]{Picture/word/media/image1.png}
\end{center}
\textbf{\textit{Trong}} \textbf{\textit{6}} 6 nước đế quốc, đế quốc nào sở hữu diện tích thuộc địa lớn nhất?\\
$\dfrac{1000}{3}\text{m}\text{}$\\“Ngôn ngữ báo chí là ngôn ngữ dùng để \_\_\_\_\_\_\_\_\_\_ tin tức thời sự trong nước và quốc tế, phản ánh chính kiến của tờ báo và dư luận quần chúng, nhằm thúc đẩy sự tiến bộ của xã hội\underline{}”\\\textit{Tất cả mọi việc xảy đến khiến tôi kinh ngạc đến mức, trong mấy phút đầu, tôi cứ đứng há mồm ra mà nhìn. Thế rồi chẳng biết từ bao giờ, tôi đã vứt chiếc máy ảnh xuống đất chạy nhào tới.\\Bóng một đứa con nít lao qua trước mặt tôi. Tôi vừa kịp nhận ra thằng Phác - thằng bé trên rừng xuống vừa nằm ngủ với tôi từ lúc nửa đêm. Thằng bé cứ chạy một mạch, sự giận dữ căng thẳng làm nó khi chạy qua không nhìn thấy tôi. Như một viên đạn trên đường lao tới đích đã nhắm, mặc cho tôi gọi nó vẫn không hề ngoảnh lại, nó chạy tiếp một quãng ngắn giữa những chiếc xe tăng rồi lập tức nhảy xổ vào cái lão đàn ông.\\Cũng y hệt người đàn bà, thằng bé của tôi cũng như một người câm, và đến lúc này tôi mới biết là nó khoẻ đến thế!\\Khi tôi chạy đến nơi thì chiếc thắt lưng da đã nằm trong tay thằng bé…\\(Nguyễn Minh Châu, Chiếc thuyền ngoài xa, NXB Giáo dục, SGK Ngữ văn 12, Tập hai, 2014)
}
\end{document}